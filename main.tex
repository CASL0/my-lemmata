\documentclass{book}
\usepackage{hyperref}
\usepackage{amsthm}
\usepackage{amssymb}

\newtheorem*{lemma}{補題}

\title{補題集}
\author{CASL0}
\date{\today}

\begin{document}
\maketitle

\frontmatter

\chapter*{はじめに}

数学書の証明を読んでいる中で、省略されたりもする論理が、筆者には理解が難しいことがしばしばあった。
それらの行間を埋める補題を覚書として残すために、本書に記載している。

数学書の命題の番号に合わせた番号を補題にも振っていくこととする。例えば、定理2.3の証明に対する補題の場合は補題2.3とする。

\tableofcontents

\mainmatter

\chapter{微分積分学、笠原}

\href{https://www.saiensu.co.jp/search/?isbn=978-4-7819-0108-4&y=1974}{微分積分学}の
証明の行間を埋める補題を記載する。

\section{集合と関数}

\begin{lemma}[1.30]
  集合A,Bおよび関数fに対して

  $A \subset B \Rightarrow f(A) \subset f(B)$

  \begin{proof}
    $\alpha \in f(A)$を任意にとる

    この時、ある$a \in A$が存在して、$\alpha = f(a)$を満たす

    一方で$A \subset B$より、$a \in B$でもある

    これは、$\alpha = f(b)$を満たす$b \in B$が存在していることを示している

    $\therefore \alpha \in f(B)$

    $\therefore f(A) \subset f(B)$
  \end{proof}
\end{lemma}

\begin{lemma}[1.32]
  集合X,Yおよび$A \subset X$上の関数fに対して

  $f(A) \subset Y \Rightarrow A \subset f^{-1}(Y)$

  \begin{proof}
    $a \in A$を任意にとる

    $f(A) \subset Y$ より、$f(a) \in Y$

    $\therefore a \in f^{-1}(Y)$

    $\therefore A \subset f^{-1}(Y)$
  \end{proof}
\end{lemma}

\begin{lemma}[1.34]
  数列$a_n$について、$\lim_{n \to \infty} a_n = a$とする

  $a_n$の部分列$a_{n(k)}$もaに収束する

  \begin{proof}
    任意の$\varepsilon  > 0$に対して、十分大きな$n_0$が存在し、$n \geq n_0 \Rightarrow |a_n
    - a| < \varepsilon $

    一方、部分列なので$n(k) \geq k$が成り立つ

    $\therefore k \geq n_0 \Rightarrow n(k) \geq n_0 \Rightarrow
    |a_{n(k)} - a| < \varepsilon $
  \end{proof}
\end{lemma}

\begin{lemma}[1.36]
  $\lim_{n \to \infty}a_n = a, \lim_{n \to \infty}b_n = b$を満たす数列$a_n,
  b_n$が与えられたとき

  数列$c_n = a_1,b_1,a_2,b_2,a_3,b_3,\dots$が収束するための必要十分条件は$a = b$である

  \begin{proof}
    (十分性)任意の$\varepsilon > 0$に対して、十分大きな$n_1, n_2$が存在し

    \[n \geq n_1 \Rightarrow |a_n - a| < \varepsilon\]
    \[n \geq n_2 \Rightarrow |b_n - a| < \varepsilon\]

    が成り立つ。
    $n_0 := \max(n_1, n_2)$とすると、$n \geq n_0 \Rightarrow |c_n - a| < \varepsilon$

    $\because n$が奇数の時は$|c_n - a| = |a_n - a| < \varepsilon$、
    $n$が偶数の時は$|c_n - a| = |b_n - a| < \varepsilon$

    (必要性)$\lim_{n \to \infty}c_n = c$とすると、補題1.34より部分列も$c$に収束する

    $\therefore \lim_{n \to \infty}a_n = \lim_{n \to \infty}b_n = c$
  \end{proof}
\end{lemma}

\end{document}
